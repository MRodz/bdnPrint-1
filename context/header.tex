% there seems to be a conflict between hyphenation and the language packages
\hyphenation{Er-zie-hungs-we-sens Schul-schrif-ten Ab-sicht The-o-phrast's}

% layout

\definelayout[odd]
  [backspace=52mm, % = (page width - text width) / 2
                   % = (210mm - 106mm) / 2
   width=106mm,
   topspace=48mm,  % = (page height - text height) / 2
                   %     - header - header distance
                   % = (297mm - 181mm) / 2 - 4mm - 6mm
   height=183mm,   % = text height + header + header distance
                   %     - negative spacing above footnotes
                   % = 181mm + 4mm + 6mm - 8mm
   header=12pt,    % =~ 4mm
   headerdistance=6mm,
   footer=0mm,
   margindistance=4mm,
   margin=20mm]

\definelayout[even]
  [backspace=52mm,
   width=106mm,
   topspace=48mm,
   height=183mm,
   header=12pt,
   headerdistance=6mm,
   footer=0mm,
   margindistance=4mm,
   margin=20mm]

\definehspace[p]
  [0.5 em]   
  
\definehspace[item]
  [-1.5 em]
\definehspace[margin-horizontal]
  [30 px]
   
\startsetups[*lessstrict]
  \setup[reset]
  \clubpenalty=1000
  \brokenpenalty=1000
  \widowpenalty=1000
\stopsetups

\setuplayout[setups=*lessstrict]



% font

\usemodule[simplefonts]

 \setmainfont[ebgaramond]

 \definefontfallback[cardoFallback]
   [name:cardo][0x0-0x10FFFF]

 \definefontsynonym[cardoSynonym]
   [Serif][fallbacks=cardoFallback]

 \definefont[cardoFont]
   [cardoSynonym]

\definefontsynonym[ezraSynonym]
  [name:ezrasil]

\definefont[ezraFont]
  [ezraSynonym]

 \definefontsynonym[ebSynonym]
   [name:ebgaramond08regular]
 
 \definefont[ebFont]
   [ebSynonym]

\definefontsynonym[termesSynonym]
  [name:termes]

\definefont[termesFont]
  [termesSynonym]
  
\definefontsynonym[linlibSynonym]
  [name: linlibertinedisplayo]
  
\definefont[linlibFont]
  [linlibSynonym]

\def\dvl{{\termesFont\char"2016}}

\def\ra{{\termesFont\char"2192}}

\def\vl{{\termesFont\char"007C}}

\def\line{{\linlibFont \char"222B}}			% 222B: integral
\def\p{{\linlibFont \char"222C}}			% 222C: double integral
\def\noline{{\linlibFont \overstrike{\char"222B}}}	% 222B: integral (struck through)
\def\nop{{\linlibFont \overstrike{\char"222C}}}		% 222C: double integral(struck through)

\definebodyfontenvironment[10pt]
  [interlinespace=12.25pt, % xx for footnote markers
   a=11.5pt, x=8.5pt, xx=7.75pt]

\definebodyfontenvironment[8.5pt]
  [interlinespace=10.5pt, % xx for footnote markers
   a=9.78pt, x=7.23pt, xx=6.59pt]

\definefont[petit]
  [Serif at 8.5pt]

\definefontfamily[garamond]
  [serif][adobegaramondpro]

\setupbodyfont
  [garamond, 10pt]

% for lining numbers instead of oldstyle numbers (default)
\definefontfeature[f:lnum] [lnum=yes] 


% header

% \setupheader[text]
%   [after=\hrule]



\setupheadertexts
  []

\startsetups[a]
  \switchtobodyfont[default]
  \rlap{}
  \hfill
  % adapt for publication: information has to be retrieved from TEI body
  {\tfx\it Philologie}
  \hfill
  \llap{\pagenumber}
\stopsetups

\startsetups[b]
  \switchtobodyfont[default]
  \rlap{\pagenumber}
  \hfill
   % adapt for publication: information has to be retrieved from TEI header
  {\tfx\it J. A. Nösselt, Anweisung zur Bildung angehender Theologen\ \high{1}1786/89–\high{3}1818/19}
  \hfill
  \llap{}
\stopsetups

\setupheadertexts
  [\setups{a}][][][\setups{b}]

\setuppagenumbering
  [alternative=doublesided]


% text

\definehspace[threeem][3em]
\definehspace[twoem][2em]

\def\emptyEvenPage{%
  \ifodd\pagenumber\page[empty]\fi%
}

\def\justifiedPageBreak{%
  \parfillskip\zeropoint\page\noindent%
}

% \setuphead[part]
%   [before={\blank[20mm, force]},
%    after={\blank[12pt]},
%    align=middle,
%    header=empty,
%    number=no,
%    placehead=yes,
%    style=\tfa]
% 
% \setuphead[section]
%   [before={\blank[24pt]},
%    after={\blank[12pt]},
%    align=middle,
%    style=\tfa]

\setuphead[part]
  [before={\blank[10mm, force]},
   after={\blank[6pt]},
   align=middle,
   header=empty,
   number=no,
   placehead=yes,
   style=\tfa]
   
\setuphead[chapter]
  [before={\blank[10mm, force]},
   after={\blank[6pt]},
   align=middle,
   header=empty,
   number=no,
   placehead=yes,
   style=\tfa]

\setuphead[section]
  [before={\blank[12pt]},
   after={\blank[6pt]},
   align=middle,
   style=\tfa]
   
\setuphead[subject]
  [before={\blank[8pt]},
   after={\blank[4pt]},
   align=middle,
   style=\tfa]
   
\setuplist[part]
  [alternative=c]
  
\setuplist[chapter] 
  [margin=1em,
  alternative=c]
  
\setuplist[section]
  [margin=2em,
  alternative=c]
  
\setuplist[subsection]
  [margin=3em,
  alternative=c]  

\setuplist[subsubsection]
  [margin=4em,
  alternative=c]    
  
\setupindenting
  [yes, 4mm, next]


\setuplanguage[de]
  [spacing=packed]

\mainlanguage[de]
\language[de]

\setupmargindata
  [style=petit]

\setupnarrower
  [left=4mm]

% TODO: optimize inter word spacing
\setuptolerance
  [verytolerant]


% editor

\definenote[editor]
  [before={\blank[4mm]},
   after={\blank[-8mm]}]

\setupnotation[editor]
  [style={\switchtobodyfont[8.5pt]}]


% registers

\setupcolumns[align=flushleft]

\defineregister[antIndex]
\setupregister[antIndex]
  [compress=yes,
   indicator=no,
   pagestyle=\tf,
   n=1]

\defineregister[bibelIndex]
\setupregister[bibelIndex]
  [compress=yes,
   indicator=no,
   pagestyle=\tf,
   n=1,
   before=\blank]

\defineregister[persIndex]
\setupregister[persIndex]
  [compress=yes,
   indicator=no,
   pagestyle=\tf,
   n=1]

\defineregister[sachIndex]
\setupregister[sachIndex]
  [compress=yes,
   indicator=no,
   pagestyle=\tf,
   n=1]
   
\defineconversion[s][,,,]

\unprotect
\unexpanded\def\startregisterentry#1% TODO: level
  {\typo_injectors_check_register
   \begingroup
   \ifnum\leftskip=\d_strc_registers_distance
     \hskip-\d_strc_registers_distance\hbox to 0mm{\endash}\hskip\d_strc_registers_distance
   \fi
   \dostarttagged\t!registerentry\empty
   \global\setconstant\c_strc_registers_page_state\zerocount
   \ifnum\leftskip=0
     \hangindent\d_strc_registers_hangindent
     \hangafter \c_strc_registers_hangafter
   \fi
   \typo_injectors_mark_register}
\protect


% lua

\def\margin#1#2#3#4#5{%
  \begingroup%
%
  \startluacode%
    local note = {
      id   = "#1";
      type = "#2";
      ref  = "#3";
      note = "#5";
    }

    note["note"] = string.gsub(note["note"], "\textbackslash", "\\textbackslash")

    notes[key] = note
    tex.setattribute(1, key)
    key = key + 1
  \stopluacode%
%
  #4%
%
  \endgroup%
}

\startluacode
key = 1
notes = {}

os.remove("id-file.txt")
os.remove("notes.txt")

idFile = io.open("id-file.txt", "a")
notesFile = io.open("notes.txt", "a")

processMargins = function(head)
  for line in node.traverse_id(node.id("hlist"), head) do
    local lineKeys = {}
    local lineNote = ""

    for item in node.traverse(line.list) do
      local itemKey = node.has_attribute(item, 1)

      if itemKey and notes[itemKey] then
        lineKeys[#lineKeys + 1] = itemKey
      end
    end

    local hash = {}
    local res = {}

    for _, v in ipairs(lineKeys) do
      if not hash[v] then
        res[#res + 1] = v
        hash[v] = true
      end
    end

    table.sort(res)
    lineKeys = res

    if lineKeys[1] then
      idFile:write(notes[lineKeys[1]]["id"])
    end

    idFile:write("\n")

    for _, k1 in ipairs(lineKeys) do
      local v1 = notes[k1]
      local backslash = false

      if v1["type"] == "omOpen" then
        for _, k2 in ipairs(lineKeys) do
          local v2 = notes[k2]

          if v2["type"] == "omClose" and v1["ref"] == v2["ref"] then
            backslash = true
          end
        end
      end

      if v1["type"] == "omClose" then
        for _, k2 in ipairs(lineKeys) do
          local v2 = notes[k2]

          if v2["type"] == "omOpen" and v1["ref"] == v2["ref"] then
            goto continue
          end
        end
      end

       if v1["type"] == "plClose" then
        for _, k2 in ipairs(lineKeys) do
          local v2 = notes[k2]

          if v2["type"] == "plOpen" and v1["ref"] == v2["ref"] then
            goto continue
          end
        end
      end

      if k1 ~= lineKeys[1] then
        lineNote = lineNote .. ", "
      end

      lineNote = lineNote .. v1["note"]

      if backslash then
        lineNote = lineNote .. "\\textbackslash"
      end

      ::continue::
    end

    notesFile:write(lineNote .. "\n")
  end

  return head
end

nodes.tasks.appendaction("finalizers", "after", "processMargins")
\stopluacode


\starttext

% for lining numbers instead of oldstyle numbers (default)
\addff{f:lnum}
